% Created 2024-10-02 Wed 21:08
% Intended LaTeX compiler: pdflatex
\documentclass[10pt,a4paper,ragged2e,withhyper]{altacv}

% Change the page layout if you need to
\geometry{left=1.25cm,right=1.25cm,top=1.5cm,bottom=1.5cm,columnsep=1.2cm}

% Use roboto and lato for fonts
\renewcommand{\familydefault}{\sfdefault}

% Change the colours if you want to
\definecolor{SlateGrey}{HTML}{2E2E2E}
\definecolor{LightGrey}{HTML}{666666}
\definecolor{DarkPastelRed}{HTML}{450808}
\definecolor{PastelRed}{HTML}{8F0D0D}
\definecolor{GoldenEarth}{HTML}{E7D192}
\colorlet{name}{black}
\colorlet{tagline}{PastelRed}
\colorlet{heading}{DarkPastelRed}
\colorlet{headingrule}{GoldenEarth}
\colorlet{subheading}{PastelRed}
\colorlet{accent}{PastelRed}
\colorlet{emphasis}{SlateGrey}
\colorlet{body}{LightGrey}

% Change some fonts, if necessary
\renewcommand{\namefont}{\Huge\rmfamily\bfseries}
\renewcommand{\personalinfofont}{\footnotesize}
\renewcommand{\cvsectionfont}{\LARGE\rmfamily\bfseries}
\renewcommand{\cvsubsectionfont}{\large\bfseries}

% Change the bullets for itemize and rating marker
% for cvskill if you want to
\renewcommand{\itemmarker}{{\small\textbullet}}
\renewcommand{\ratingmarker}{\faCircle}

\usepackage[rm]{roboto}
\usepackage[defaultsans]{lato}
\usepackage{paracol}
\columnratio{1.075} % Set the left/right column width ratio to 6:4.
\usepackage[bottom]{footmisc}
\DeclareNameAlias{sortname}{given-family}
\addbibresource{aidan.bib}
\usepackage[style=trad-abbrv,sorting=none,sortcites=true,doi=false,url=false,giveninits=true,hyperref]{biblatex}
\author{Aidan Scannell}
\date{\today}
\title{}
\hypersetup{
 pdfauthor={Aidan Scannell},
 pdftitle={},
 pdfkeywords={},
 pdfsubject={},
 pdfcreator={Emacs 27.2 (Org mode 9.4.6)}, 
 pdflang={English}}
\begin{document}

\name{Gaurav Sood}
\photoR{2.8cm}{gaurav.jpeg}
\tagline{Sr. Data Scientist}

\personalinfo{
  \email{gsood.gaurav@gmail.com}
  \phone{+91 9632714987}
  \location{Bangalore, India}
  \github{github.com/gsood-gaurav}
  \linkedin{linkedin.com/in/gsood-gaurav/}
}
\makecvheader

\begin{paracol}{2}
 \begin{quote}
"Senior Data Scientist with total 14 years of experience, and 7 years of experience in Machine Learning, developing open source software in Python, C++, contributing as both individual contributor and team leader, working on Machine Learning, conceptualizing, designing and implementing end to end solutions. Currently working on ideas around Natural Language Processing and Reinforcement Learning to automate Hardware Diagnostics. Have used both traditional machine learning methods like Bayesian Networks, Conditional Random Fields and modern deep learning methods LSTMs, Transformer based models like BERT and GPT, using TensorFlow Recently started using Julia and Flux for Machine Learning.
"
 \end{quote}
\cvsection{Skills}
\label{sec:org980d01b}
\cvtag{Python}
\cvtag{PyTorch}
\cvtag{TensorFlow}
\cvtag{JAX}
\cvtag{Julia}
\cvtag{Flux}
\cvtag{NumPy}
\cvtag{SciPy}
\cvtag{Matplotlib}


\divider

\cvtag{Deep Learning}
\cvtag{Large Language Models}
\cvtag{Generative AI}
\cvtag{Probabilistic Programming}
\cvtag{Reinforcement Learning}
\cvtag{Open Source Software}

\cvsection{Experience}
\label{sec:orge98f48b}
\cvevent{Sr. Data Scientist}{ HP Inc. June 2021 -- Ongoing}{ Bangalore, India}{}

\begin{itemize}
\item Designed and Developed Large Language Models Based application for web content filtering/selection.
\item Mentored three data science interns on Hardware logs analysis using Graph
Neural Networks and Large Language Models.
\item Fine Tuned Small language Chat model on custom data using 4-bit quantized models.
\item Developed Telemetery data analysis pipline using Large Language Model
embeddings, UMAP, Variational AutoEncoder.
\item 
\end{itemize}
\cvtag{Probabilistic modelling}
\cvtag{Gaussian processes}
\cvtag{Variational inference}
\cvtag{Optimal control}
\cvtag{Trajectory optimisation}

\par\divider
\cvevent{Software Engineer Machine Learning}{ HP Inc. June 2017 -- 2021}{ Bangalore, India}{}

\begin{itemize}
\item Analyzed Sales Time Series Data (Short term/Long term forecasting) using
Prophet, ARIMA, Gradient Boosting models.
\item Behavior Modelling using Tree Based Models, SHAP, UMAP
\item Proposed methodology which specifies automated hardware diagnostics problem
in Reinforcement Learning Framework.
\item Design and implemented software module which models environment for
Reinforcement Learning agent to learn using Monte Carlo Methods.
\item Worked on Conditional Random Fields for Sequence Tagging and Information
extraction.
\item 
\end{itemize}
\cvtag{Communication}
\cvtag{Active listening}
\cvtag{Teaching}

\begin{enumerate}
\item Senior Software Engineer
\label{sec:org5019653}
\par\divider
\cvevent{Senior Software Engineer}{ HP Inc. June 2015 -- 2017}{ Bangalore, India}{}
\begin{itemize}
\item Designed and Developed a parser for PostScript Language to extract images and
meta data which was impacting 50000 hp customers and saving 5 million dollars.
\item Designed and developed Telemetery module in python.
\item Led team of 7 software developers designing and implementing open source
software used by over 1 million customers.
\item Worked on understanding PDF specification, design and implemented PDF document
processing using PDF libraries like iText and PDFium
\item Lead a team of six people which delived Imaging Processing Solutions to over
10 PSU banks in india
\end{itemize}

\cvtag{Engineering}
\cvtag{Teamwork}
\cvtag{Industry}

\item Software Engineer
\label{sec:org92e244d}
\par\divider
\cvevent{Software Engineer}{ HP Inc. June 2010 -- 2015}{ Bangalore, India}{}
\end{enumerate}

\cvsection{Certifications}
\label{sec:orgec06d61}
\cvevent{Deep Learning Specialization}{ Coursera Oct-2021 Credential ID G8PG2GY6WFS)}{}{}
\cvevent{Reinforcement Learning Specialization}{ Coursera Sept-2021 Credential IDD3LDNU8BRW68}{}{}
\cvevent{Introduction to Quantum Computing}{ Coursera May-2021 Credential ID RXQHLEC2WYC8}{}{}
\nocite{*}
% \printbibliography[heading=pubtype,title={\printinfo{\faBook}{Books}},type=book]
% \divider
% \printbibliography[heading=pubtype,title={\printinfo{\faFile*[regular]}{Journal Articles}},type=article]
% \divider
\printbibliography[heading=pubtype,title={\printinfo{\faUsers}{Conference Proceedings}},type=inproceedings]

\cvsection{Projects}
\label{sec:org8999906}
\cvevent{Optimal Control in Multimodal Dynamical Systems as Probabilistic Inference}{ University of Bristol}{ May 2021 - Ongoing}{ Bristol, UK}

\begin{itemize}
\item Developing data-efficient techniques for exploration in multimodal dynamical systems.
\item The goal of this project is to explore a single dynamics mode that is known to be operatable whilst avoiding other modes.
\end{itemize}

\cvtag{Variational inference}
\cvtag{Gaussian processes}
\cvtag{Optimal control}

\newpage

\cvsection{Projects (Cont.)}
\label{sec:org8a43f82}
\cvevent{Trajectory Optimisation in Learned Multimodal Dynamical Systems}{ University of Bristol}{ Sept 2019 - March 2021}{ Bristol, UK}

\begin{itemize}
\item Synergising Bayesian inference and Riemannian geometry to control multimodal dynamical systems.
\item Finds trajectories that 1) remain in a desired dynamics mode, 2) avoid regions of the dynamics with high epistemic uncertainty.
\item \href{https://github.com/aidanscannell/trajectory-optimisation-in-learned-multimodal-dynamical-systems}{\faGithub aidanscannell/trajectory-optimisation-in-learned-multimodal-dynamical-systems}
\end{itemize}

\cvtag{JAX}
\cvtag{Probabilistic geometries}
\cvtag{Optimal control}
\par\divider

\cvevent{Identifiable Mixtures of Sparse Variational Gaussian Process Experts}{ University of Bristol}{ Sept 2018 - Ongoing}{ Bristol, UK}

\begin{itemize}
\item Improving identifiability and scalability in the Mixtures of Gaussian Process Experts model with GP-based gating networks.
\item Variational inference based on sparse GP approximations.
\item \href{https://github.com/aidanscannell/mogpe}{\faGithub aidanscannell/mogpe}
\end{itemize}

\cvtag{GPflow}
\cvtag{TensorFlow}
\cvtag{Gaussian processes}
\cvtag{Variational inference}

\cvsection{Volunteering}
\label{sec:orgee03d65}
\cvevent{Cohort Representative}{ FARSCOPE CDT}{ Sept 2018 - Ongoing}{ Bristol, UK}

\begin{itemize}
\item Represent myself and my CDT peers in management meetings.
\item Communicate information between students and management.
\end{itemize}

\cvtag{Communication}
\cvtag{Interpersonal Skills}

\par\divider
\cvevent{Club Leader}{ Code Club}{ Dec 2017 - April 2018}{ Junction 3 Library, Bristol , UK}

\begin{itemize}
\item Set up (and then ran) a \href{https://codeclub.org/en/}{Code Club} for children aged 9-13.
\item Led the organisation, planning and teaching of weekly lessons.
\item Planned lessons to engage children by making coding fun.
\item Extremely rewarding and reinforced my love for teaching.
\end{itemize}

\cvtag{Leadership}
\cvtag{Teaching}
\cvtag{Communication}
\cvtag{Active listening}

\cvsection{Invited Talks}
\label{sec:orgb9bfa3f}
\cvevent{Synergising Bayesian Inference and Probabilistic Geometries for Robotic Control}{ Cognitive Systems - Technical University of Denmark (DTU)}{ 18 March 2021}{ Zoom}

\begin{itemize}
\item Presented a method synergising Bayesian inference and probabilistic geometries to control multimodal dynamical systems.
\end{itemize}

\cvtag{Communication}
\cvtag{Probabilistic geometries}
\cvtag{Gaussian processes}


\cvsection{Education}
\label{sec:org872d601}
\cvevent{Msc Research \ Speech Recoginition}{ Indian Institute of Science Bangalore 2006-2009}{}{}
\begin{itemize}
\item \faBook Speech Recognition using Support Vector Machines
\end{itemize}

\divider

\cvevent{BTech Electronics and Communication Engg}{ GNE Ludhiana 2001-2005}{}{}

\divider

\cvevent{Machine Learning Summer School Moscow (MLSS)}{ Skoltech}{ Aug 2019 - Sept 2019}{}

\divider

\cvevent{\footnote{Awarded if PhD is not completed.} MRes in Robotics \& Autonomous Systems}{University of Bristol | First Class Honours}{Sept 2017 -- Sept 2018}{}

\begin{itemize}
\item \faBook \href{https://www.aidanscannell.com/project/uncertain-agentspeak/}{Extending BDI Agents to Model and Reason with Uncertainty}
\end{itemize}


\divider

\cvevent{MEng in Mechanical Engineering}{ University of Bristol | First Class Honours}{ Sept 2012 -- June 2016}{}

\begin{itemize}
\item Graduated in top 10\% of cohort
\end{itemize}

\newpage

\cvsection{Achievements}
\label{sec:orgf001db1}
\cvachievement{\faTrophy}{ Full Sporting Colours}{ Awarded full colours for outstanding achievements in snowboarding. Multiple gold medals in British University Snowboard Championships.}{}

\divider

\cvachievement{\faCertificate}{ Starting To Teach}{ Established myself as a confident, enthusiastic and effective teacher who is able to engage, encourage and develop students' learning.}{}

\divider

\cvachievement{\faTrophy}{Bristol Plus Award}{ For undertaking a wide range of tasks to further enhance student skills - only 700 out of 23,000 achieved this award per annum.}{}

\divider

\cvachievement{\faCertificate}{ Mary Jones Prize for Mathematics}{ For outstanding achievements in A Level mathematics @ Ripon Grammar School}{}

\divider

\cvachievement{\faTrophy}{ The Duke of Edinburgh's Award}{ Bronze/Silver/Gold}{}

\cvsection{References}
\label{sec:org1c639c7}
% \cvref{name}{email}{mailing address}
\cvref{Prof.\ Arthur Richards}{University of Bristol}{arthur.richards@bristol.ac.uk}
% {Address Line 1\\Address line 2}
\divider
\cvref{Dr.\ Carl Henrik Ek}{University of Cambridge}{che29@cam.ac.uk}
% {Address Line 1\\Address line 2}
\end{paracol}
\end{document}
\end{document}
