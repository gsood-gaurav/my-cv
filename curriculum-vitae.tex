% Created 2024-10-27 Sun 20:43
% Intended LaTeX compiler: pdflatex
\documentclass[10pt,a4paper,ragged2e,withhyper]{altacv}

% Change the page layout if you need to
\geometry{left=1.25cm,right=1.25cm,top=1.5cm,bottom=1.5cm,columnsep=1.2cm}

% Use roboto and lato for fonts
\renewcommand{\familydefault}{\sfdefault}

% Change the colours if you want to
\definecolor{SlateGrey}{HTML}{2E2E2E}
\definecolor{LightGrey}{HTML}{666666}
\definecolor{DarkPastelRed}{HTML}{450808}
\definecolor{PastelRed}{HTML}{8F0D0D}
\definecolor{GoldenEarth}{HTML}{E7D192}
\colorlet{name}{black}
\colorlet{tagline}{PastelRed}
\colorlet{heading}{DarkPastelRed}
\colorlet{headingrule}{GoldenEarth}
\colorlet{subheading}{PastelRed}
\colorlet{accent}{PastelRed}
\colorlet{emphasis}{SlateGrey}
\colorlet{body}{LightGrey}

% Change some fonts, if necessary
\renewcommand{\namefont}{\Huge\rmfamily\bfseries}
\renewcommand{\personalinfofont}{\footnotesize}
\renewcommand{\cvsectionfont}{\LARGE\rmfamily\bfseries}
\renewcommand{\cvsubsectionfont}{\large\bfseries}

% Change the bullets for itemize and rating marker
% for cvskill if you want to
\renewcommand{\itemmarker}{{\small\textbullet}}
\renewcommand{\ratingmarker}{\faCircle}

\usepackage[rm]{roboto}
\usepackage[defaultsans]{lato}
\usepackage{paracol}
\usepackage[bottom]{footmisc}
\DeclareNameAlias{sortname}{given-family}
\addbibresource{aidan.bib}
\usepackage[style=trad-abbrv,sorting=none,sortcites=true,doi=false,url=false,giveninits=true,hyperref]{biblatex}
\author{Gaurav Sood}
\date{\today}
\title{}
\hypersetup{
 pdfauthor={Gaurav Sood},
 pdftitle={},
 pdfkeywords={},
 pdfsubject={},
 pdfcreator={Emacs 27.2 (Org mode 9.4.6)}, 
 pdflang={English}}
\begin{document}

\name{Gaurav Sood}
\photoR{2.8cm}{gaurav.jpeg}
\tagline{Sr. Data Scientist}

\personalinfo{
  \email{gsood.gaurav@gmail.com}
  \phone{+91 9632714987}
  \location{Bangalore, India}
  \github{github.com/gsood-gaurav}
  \linkedin{linkedin.com/in/gsood-gaurav/}
}
\makecvheader

\begin{paracol}{1}
 \begin{quote}
"Senior Data Scientist with 7 years of experience in Data Science and total experience of 14 years in software industry, contributed as an individual contributor and team leader, building machine learning models from conception to completion to improve operational efficiency. Worked on  Natural Language Processing, Time Series Forecasting, Tabular Datasets. Experienced in traditional machine learning methods (Probabilistic Graphical Models), deep learning models (LSTMs, CNNs), Transformer based models (BERT, GPT and LLMs), Graph Neural Networks and Reinforcement Learning"
 \end{quote}
\cvsection{Skills}
\label{sec:org8c5fe86}
\cvtag{Python}
\cvtag{PyTorch}
\cvtag{TensorFlow}
\cvtag{JAX}
\cvtag{Julia}
\cvtag{Flux/Lux}
\cvtag{NumPy}
\cvtag{SciPy}
\cvtag{Matplotlib}


\divider

\cvtag{Large Language Models}
\cvtag{Generative AI}
\cvtag{Probabilistic Modelling}
\cvtag{Reinforcement Learning}
\cvtag{Open Source Software}
\divider

\cvtag{Communication}
\cvtag{Leadership Skills}

\cvsection{Experience}
\label{sec:orgd0f7e4d}
\cvevent{Sr. Data Scientist}{ HP Inc. May 2021 -- Ongoing}{ Bangalore, India}{}

\begin{itemize}
\item Created Large Language Models(LLMs) Based application for webpage content
filtering and topic recommendation, adhering to key NFRs which helped in
acheiving \textbf{\textbf{Net Promoter Score of 65}} during beta testing of HP's PrintAI
program.
\item Benchmarked LLM based application's performance on accuracy, cost(token usage)
and latency, comparing OpenAI, Llama and Mistral, reducing cost by \textbf{\textbf{30\%}}.
\item Mentored \textbf{\textbf{3}} data science interns on automated triaging using Graph Neural
Networks and LLMs.
\item Supervised Small Language Models(SLMs, less than \textbf{\textbf{3B}} parameters, \textbf{\textbf{4bit}}
quantized) exploration and \textbf{\textbf{instruction finetuning}}
initatives using PEFT acheiving an accuracy of 95\%
\item Created Telemetery data analysis pipline processing \textbf{\textbf{0.5}} milllion customer
record using LLM embeddings, BERT, dimensionality reduction(UMAP, Variational
AutoEncoder) and clustering HDBScan, enhancing operational efficiency by \textbf{\textbf{20\%-30\%}}.
\end{itemize}

\cvtag{Prompt Engineering}
\cvtag{GPT2}
\cvtag{TinyLlama}
\cvtag{FineTuning}
\cvtag{Embedding Models}
\cvtag{Graph Nerual Networks}
\par\divider
\cvevent{Sr. Software Engineer Machine Learning}{ HP Inc. July 2017 -- May 2021}{ Bangalore, India}{}

\begin{itemize}
\item Analyzed the effects of key macro economic indicators on long and short
term forecasts of product sales data which helped in acheiving on-time
delivery rate of \textbf{\textbf{90-95\%}}
\item Implemented Hyperparameter Tuning module for Time Series models using Optuna
reducing the AutoML framework execution time by \textbf{\textbf{20\%-25\%}}
\item Worked on HP Personal Systems User Behavior Modelling using Tree Based Models,
SHAP, UMAP, HDBScan for targeted market campaigning
\item Developed simulator environment for online Reinforcement Learning for
automated hardware diagnostics
\item Mentored \textbf{\textbf{4}} interns on Sequence Tagging Models using BERT, LSTMs,
Conditional Random Fields for discovering insights in call center records
which lead to \textbf{\textbf{25\%-30\%}} increase in operational efficency.
\item Presented and communicated ideas about Probabilistic Modelling and
Reinforcement Learning to senior leadership
\item Analyzed \textbf{\textbf{0.1 million}} call center record using Bayesian Networks for predictive
diagnostics as an aid to call center executives and evaluate their decision
making process
\end{itemize}

\cvtag{Time Series Analysis}
\cvtag{CART}
\cvtag{Reinforcement Learning}
\cvtag{Probabilistic Graphical Models}
\cvtag{Data Annotation}

\par\divider
\cvevent{Senior Software Engineer}{ HP Inc. March 2015 -- June 2017}{ Bangalore, India}{}
\begin{itemize}
\item Created a parser for PostScript Language to extract images and
meta data which was impacting \textbf{\textbf{50000}} hp customers and saving \textbf{\textbf{\$0.5 million}}
\item Worked on understanding PDF specification, design and implemented PDF document
processing using PDF libraries like iText and PDFium which helped in saving
annual licensing cost of \textbf{\textbf{\$10,000}}
\item Lead a team of \textbf{\textbf{4}} software engineers which delivered imaging processing solutions using
Python Imaging Library, Tessaract OCR to over \textbf{\textbf{10 Public sector banks across India}}
\end{itemize}

\cvtag{Algorithms}
\cvtag{OpenSource Sofware}
\cvtag{Image Processing}
\cvtag{Leadership Skills}
\par\divider
\cvevent{Software Engineer}{ HP Inc. May 2010 -- Feb 2015}{ Bangalore, India}{}
\begin{itemize}
\item Designed and Implemented Telemetry module in python using asyncio which
enabled the collection of data across \textbf{\textbf{1 millon}} devices used by hp customers
\item Collabrated with \textbf{\textbf{7}} software engineers on open source software used by \textbf{\textbf{1}}
million customers worldwide (\url{https://developers.hp.com/hp-linux-imaging-and-printing})
\end{itemize}

\cvtag{Python}
\cvtag{Open Source Software}
\cvtag{Asynchronous Programming}

\cvsection{Preventive Disclosures}
\label{sec:org39ce978}
\begin{itemize}
\item Gaurav Sood, Ranjith Kumar Gopa, Nandan Kumar Nidhi(May 2024) "Representing
Tabular Data using Large Lanugage Models"
\url{https://www.tdcommons.org/dpubs\_series/6998/}
\item Gaurav Sood (Feb 2021) "A Reinforcement Learning Approach to Printer Diagnostics"
\url{https://www.tdcommons.org/dpubs\_series/4244/}
\item Gaurav	Sood, Pavleen Kaur Brar, Raghu Anantharangachar (April 2019)
"Sequence Prediction for Print/Scan Issues using Bayesian Networks"
\url{https://www.tdcommons.org/dpubs\_series/2117/}
\end{itemize}

\cvsection{Certifications}
\label{sec:org489552c}
\cvevent{Deep Learning Specialization}{ Coursera Oct-2021 Credential ID G8PG2GY6WFS)}{}{}
\cvevent{Reinforcement Learning Specialization}{ Coursera Sept-2021 Credential IDD3LDNU8BRW68}{}{}
\cvevent{Introduction to Quantum Computing}{ Coursera May-2021 Credential ID RXQHLEC2WYC8}{}{}
\nocite{*}
% \printbibliography[heading=pubtype,title={\printinfo{\faBook}{Books}},type=book]
% \divider
% \printbibliography[heading=pubtype,title={\printinfo{\faFile*[regular]}{Journal Articles}},type=article]
% \divider
\printbibliography[heading=pubtype,title={\printinfo{\faUsers}{Conference Proceedings}},type=inproceedings]

\cvsection{Education}
\label{sec:org199f08a}
\cvevent{MSc Research \ Speech Recoginition}{ Indian Institute of Science Bangalore 2006-2009}{}{}
\begin{itemize}
\item \faBook Speech Recognition using Support Vector Machines
\end{itemize}

\divider

\cvevent{BTech Electronics and Communication Engg}{ GNE Ludhiana 2001-2005}{}{}

\end{paracol}
\end{document}
\end{document}
