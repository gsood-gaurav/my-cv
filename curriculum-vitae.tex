% Created 2024-10-10 Thu 13:55
% Intended LaTeX compiler: pdflatex
\documentclass[10pt,a4paper,ragged2e,withhyper]{altacv}

% Change the page layout if you need to
\geometry{left=1.25cm,right=1.25cm,top=1.5cm,bottom=1.5cm,columnsep=1.2cm}

% Use roboto and lato for fonts
\renewcommand{\familydefault}{\sfdefault}

% Change the colours if you want to
\definecolor{SlateGrey}{HTML}{2E2E2E}
\definecolor{LightGrey}{HTML}{666666}
\definecolor{DarkPastelRed}{HTML}{450808}
\definecolor{PastelRed}{HTML}{8F0D0D}
\definecolor{GoldenEarth}{HTML}{E7D192}
\colorlet{name}{black}
\colorlet{tagline}{PastelRed}
\colorlet{heading}{DarkPastelRed}
\colorlet{headingrule}{GoldenEarth}
\colorlet{subheading}{PastelRed}
\colorlet{accent}{PastelRed}
\colorlet{emphasis}{SlateGrey}
\colorlet{body}{LightGrey}

% Change some fonts, if necessary
\renewcommand{\namefont}{\Huge\rmfamily\bfseries}
\renewcommand{\personalinfofont}{\footnotesize}
\renewcommand{\cvsectionfont}{\LARGE\rmfamily\bfseries}
\renewcommand{\cvsubsectionfont}{\large\bfseries}

% Change the bullets for itemize and rating marker
% for cvskill if you want to
\renewcommand{\itemmarker}{{\small\textbullet}}
\renewcommand{\ratingmarker}{\faCircle}

\usepackage[rm]{roboto}
\usepackage[defaultsans]{lato}
\usepackage{paracol}
\usepackage[bottom]{footmisc}
\DeclareNameAlias{sortname}{given-family}
\addbibresource{aidan.bib}
\usepackage[style=trad-abbrv,sorting=none,sortcites=true,doi=false,url=false,giveninits=true,hyperref]{biblatex}
\author{Gaurav Sood}
\date{\today}
\title{}
\hypersetup{
 pdfauthor={Gaurav Sood},
 pdftitle={},
 pdfkeywords={},
 pdfsubject={},
 pdfcreator={Emacs 27.2 (Org mode 9.4.6)}, 
 pdflang={English}}
\begin{document}

\name{Gaurav Sood}
\photoR{2.8cm}{gaurav.jpeg}
\tagline{Sr. Data Scientist}

\personalinfo{
  \email{gsood.gaurav@gmail.com}
  \phone{+91 9632714987}
  \location{Bangalore, India}
  \github{github.com/gsood-gaurav}
  \linkedin{linkedin.com/in/gsood-gaurav/}
}
\makecvheader

\begin{paracol}{1}
 \begin{quote}
"Senior Data Scientist with total 14 years of experience, and 7 years of experience in Machine Learning, developing open source software in Python, C++, contributing as both individual contributor and team leader, working on Machine Learning, conceptualizing, designing and implementing end to end solutions. Currently working on  Natural Language Processing and Reinforcement Learning. Have used both traditional machine learning methods like Bayesian Networks, Conditional Random Fields and modern deep learning methods LSTMs, Transformer based models like BERT and GPT"
 \end{quote}
\cvsection{Skills}
\label{sec:org2540a44}
\cvtag{Python}
\cvtag{PyTorch}
\cvtag{TensorFlow}
\cvtag{JAX}
\cvtag{Julia}
\cvtag{Flux}
\cvtag{NumPy}
\cvtag{SciPy}
\cvtag{Matplotlib}


\divider

\cvtag{Large Language Models}
\cvtag{Generative AI}
\cvtag{Probabilistic Modelling}
\cvtag{Reinforcement Learning}
\cvtag{Open Source Software}
\divider

\cvtag{Communication}
\cvtag{Leadership Skills}

\cvsection{Experience}
\label{sec:org08bb26d}
\cvevent{Sr. Data Scientist}{ HP Inc. June 2021 -- Ongoing}{ Bangalore, India}{}

\begin{itemize}
\item Designed and Developed Large Language Models Based application for web content
filtering acheiving a recall of 0.94
\item Implemented evaluation scripts to benchmark LLM application's performance on
accuracy, cost(token usage) and inference time comparing GPT4/GPT3.5/GPT-4o-mini/GPT-4o
\item Mentored three data science interns on Hardware logs analysis using Graph
Neural Networks and Large Language Models(GPT-2) achieving precision/recall of
0.85/0.73 capturing structural, sequetial, semantic relationship among
different components generating logs
\item Fine Tuned 4-bit quantized Small language Chat model TinyLLama(1B parameters) on custom data set using QLORA
acheiving an accuracy of 0.96
\item Developed Telemetery data analysis pipline using Large Language Model
embeddings, sentence transformers, UMAP, Variational AutoEncoder and HDBScan
\end{itemize}

\cvtag{Prompt Engineering}
\cvtag{GPT2}
\cvtag{FineTuning}
\cvtag{Probabilistic Modelling}
\cvtag{Leadership Skills}
\cvtag{Graph Nerual Networks}
\cvtag{Embedding Models}
\par\divider
\cvevent{Software Engineer Machine Learning}{ HP Inc. June 2017 -- 2021}{ Bangalore, India}{}

\begin{itemize}
\item Analyzed Sales Time Series Data (Short term/Long term forecasting) using
Prophet, ARIMA, Gradient Boosting models, studying the effects of Consumer
Price Index, WholeSale Price Index, Industrial Production Growth and
employment, hyperparamter tuning using Optuna, producing a MAPE score of 6-8\%.
\item Worked on HP Personal Systems User Behavior Modelling using Tree Based Models,
SHAP, UMAP, HDBScan for targeted market campaigning.
\item Developed simulator environment for online Reinforcement Learning for
automated hardware diagnostics.
\item Developed Sequence Tagging Model using LSTMs for discovering insights in
call center records acheiving a precision and recall score of 0.91/0.85
\item Mentored 4 interns on sequence tagging using Conditional Random Fields to
extract insights from call center records.
\item Presented and communicated ideas about Probabilistic Modelling and
Reinforecement Learning to senior leadership
\item Analyzed Call center record using Bayesian Networks for predictive diagnostics
as an aid to call center executives.
\end{itemize}

\cvtag{Time Series Analysis}
\cvtag{CART}
\cvtag{Reinforcement Learning}
\cvtag{Probabilistic Graphical Models}
\cvtag{Data Annotation}

\newpage
\par\divider
\cvevent{Senior Software Engineer}{ HP Inc. June 2015 -- 2017}{ Bangalore, India}{}
\begin{itemize}
\item Designed and Developed a parser for PostScript Language to extract images and
meta data which was impacting 50000 hp customers and saving 5 million dollars.
\item Designed and developed Telemetery module in python.
\item Led team of 7 software developers designing and implementing open source
software used by over 1 million customers.
\item Worked on understanding PDF specification, design and implemented PDF document
processing using PDF libraries like iText and PDFium
\item Lead a team of six people which delived Imaging Processing Solutions using
Python Imaging Library, TessaractOCT to over 10 PSU banks in india
\end{itemize}

\cvtag{Algorithms}
\cvtag{OpenSource Sofware}
\cvtag{Image Processing}
\cvtag{Leadership Skills}
\par\divider
\cvevent{Software Engineer}{ HP Inc. June 2010 -- 2015}{ Bangalore, India}{}
\begin{itemize}
\item Designed and Implemented Telemetry module in python using asyncio.
\item Led team of 7 software developers designing and implementing open source
software used by over 1 million customers.
\item Collaborated with customers and stakeholders, gathering requirements and
communicating constraints.
\item Developed UI, backend, modules in Python for the hp's open source imaging
solution (\url{https://developers.hp.com/hp-linux-imaging-and-printing})
\end{itemize}

\cvtag{Python}
\cvtag{Open Source Software}
\cvtag{Asynchronous Programming}

\cvsection{Certifications}
\label{sec:orgba00798}
\cvevent{Deep Learning Specialization}{ Coursera Oct-2021 Credential ID G8PG2GY6WFS)}{}{}
\cvevent{Reinforcement Learning Specialization}{ Coursera Sept-2021 Credential IDD3LDNU8BRW68}{}{}
\cvevent{Introduction to Quantum Computing}{ Coursera May-2021 Credential ID RXQHLEC2WYC8}{}{}
\nocite{*}
% \printbibliography[heading=pubtype,title={\printinfo{\faBook}{Books}},type=book]
% \divider
% \printbibliography[heading=pubtype,title={\printinfo{\faFile*[regular]}{Journal Articles}},type=article]
% \divider
\printbibliography[heading=pubtype,title={\printinfo{\faUsers}{Conference Proceedings}},type=inproceedings]

\cvsection{Preventive Disclosures}
\label{sec:orga32ad01}
\begin{itemize}
\item Gaurav Sood, Ranjith Kumar Gopa, Nandan Kumar Nidhi(May 2024) "Representing
Tabular Data using Large Lanugage Models"
\url{https://www.tdcommons.org/dpubs\_series/6998/}
\item Gaurav Sood (Feb 2021) "A Reinforcement Learning Approach to Printer Diagnostics"
\url{https://www.tdcommons.org/dpubs\_series/4244/}
\item Gaurav	Sood, Pavleen Kaur Brar, Raghu Anantharangachar (April 2019)
"Sequence Prediction for Print/Scan Issues using Bayesian Networks"
\url{https://www.tdcommons.org/dpubs\_series/2117/}
\end{itemize}

\cvsection{Education}
\label{sec:org5cbbe08}
\cvevent{Msc Research \ Speech Recoginition}{ Indian Institute of Science Bangalore 2006-2009}{}{}
\begin{itemize}
\item \faBook Speech Recognition using Support Vector Machines
\end{itemize}

\divider

\cvevent{BTech Electronics and Communication Engg}{ GNE Ludhiana 2001-2005}{}{}

\end{paracol}
\end{document}
\end{document}
